\documentclass[modern]{aastex631}
\usepackage{showyourwork}

\newcommand{\project}[1]{\textsf{#1}}

% \shorttitle{AASTeX v6.3.1 Sample article}
\shortauthors{Foreman-Mackey et al.}

\begin{document}

\title{A general framework for scalable Gaussian Processes with applications to astronomical time series}

\author[0000-0002-9328-5652]{Daniel Foreman-Mackey}
\affiliation{Center for Computational Astrophysics, Flatiron Institute, New York, NY, USA}

\author[0000-0002-0802-9145]{Eric Agol}
\affiliation{Department of Astronomy, University of Washington, Seattle, WA, USA}
\affiliation{Virtual Planetary Laboratory, University of Washington, Seattle, WA, USA}

\author[0000-0001-5253-1987]{Tyler A.\ Gordon}
\affiliation{Department of Astronomy, University of Washington, Seattle, WA, USA}

\begin{abstract}
    In this paper we describe some generalizations to the \project{celerite} method.
\end{abstract}

\keywords{Astrostatistics (1882) --- Algorithms (1883) --- Time series analysis (1916)}

\section{Introduction}
\label{sec:intro}

This manuscript was prepared with \project{showyourwork}\footnote{https://show-your.work} \citep{Luger:2021}.

\section{Quasiseparable linear algebra}

\section{The \project{celerite} matrices}

\section{Multi-band time series}

With non-simultaneous observations.

\section{Derivative models}

\section{Time-integrated models}



\begin{acknowledgments}
    DFM would like to thank the Astronomical Data Group at the Flatiron Institute for listening to every iteration of this project and for providing great feedback every step of the way.
    The authors would also like to thank
    Suzanne Aigrain (Oxford),
    Gregory Guida (Etsy),
    Christopher Stumm (Etsy),
    \emph{and others\ldots}
    for insightful conversations and other contributions.
\end{acknowledgments}

\software{showyourwork \citep{Luger:2021}, JAX \citep{Bradbury:2018}}

\appendix
\section{Algorithms}


\bibliography{bib}
\bibliographystyle{aasjournal}

\end{document}
