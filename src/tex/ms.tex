\documentclass[modern]{aastex631}
\usepackage{showyourwork}

\newcommand{\project}[1]{\textsf{#1}}
\newcommand{\mv}[1]{\ensuremath{\boldsymbol{#1}}}
\newcommand{\mm}[1]{\ensuremath{\boldsymbol{#1}}}
\newcommand{\T}{\ensuremath{^\mathrm{T}}}

% \shorttitle{AASTeX v6.3.1 Sample article}
\shortauthors{Foreman-Mackey et al.}

\begin{document}

\title{A general framework for scalable Gaussian Processes with applications to astronomical time series}

\author[0000-0002-9328-5652]{Daniel Foreman-Mackey}
\affiliation{Center for Computational Astrophysics, Flatiron Institute, New York, NY, USA}

\author[0000-0002-0802-9145]{Eric Agol}
\affiliation{Department of Astronomy, University of Washington, Seattle, WA, USA}
\affiliation{Virtual Planetary Laboratory, University of Washington, Seattle, WA, USA}

\author[0000-0001-5253-1987]{Tyler A.\ Gordon}
\affiliation{Department of Astronomy, University of Washington, Seattle, WA, USA}

\begin{abstract}
    In this paper we describe some generalizations to the \project{celerite} method.
\end{abstract}

\keywords{Astrostatistics (1882) --- Algorithms (1883) --- Time series analysis (1916)}

\section{Introduction}
\label{sec:intro}

This manuscript was prepared with \project{showyourwork}\footnote{\url{https://show-your.work}} \citep{Luger:2021}.

\citep{Foreman-Mackey:2017, Foreman-Mackey:2018, Foreman-Mackey:2021}.

\section{Quasiseparable linear algebra}

There exist a range of definitions for \emph{quasiseparable matrices} in the literature, so to be explicit, let's select the one that we will consider in all that follows.
The most suitable definition for our purposes is nearly identical to the one used by \citet{Eidelman:1999}, with some small modifications that will become clear as we go.

Let's start by considering an $N \times N$ \emph{square quasiseparable matrix} $\mm{M}$ with lower quasiseparable order $m_l$ and upper quasiseparable order $m_u$.
In this note, we represent this matrix $\mm{M}$ as:\footnote{Comparing this definition to the one from \citet{Eidelman:1999}, you may notice that we have swapped the labels of $\mv{g}_j$ and $\mv{h}_i$, and that we've added an explicit transpose to $\mm{B}_k\T$. These changes simplify the notation and implementation for symmetric matrices where, with our definition, $\mv{g} = \mv{p}$, $\mv{h} = \mv{q}$, and $\mm{B} = \mm{A}$.}
%
\begin{eqnarray}\label{eq:square-qsm-def}
    M_{ij} &=& \left \{ \begin{array}{ll}
        d_i\quad,                                                                   & \mbox{if }\, i = j \\
        \mv{p}_i\T\,\left ( \prod_{k=i-1}^{j+1} \mm{A}_k \right )\,\mv{q}_j\quad,   & \mbox{if }\, i > j \\
        \mv{h}_i\T\,\left ( \prod_{k=i+1}^{j-1} \mm{B}_k\T \right )\,\mv{g}_j\quad, & \mbox{if }\, i < j \\
    \end{array}\right .
\end{eqnarray}
%
where
%
\begin{itemize}
    \item $i$ and $j$ both range from $1$ to $N$,
    \item $d_i$ is a scalar,
    \item $\mv{p}_i$ and $\mv{q}_j$ are both vectors with $m_l$ elements,
    \item $\mm{A}_k$ is an $m_l \times m_l$ matrix,
    \item $\mv{g}_j$ and $\mv{h}_i$ are both vectors with $m_u$ elements, and
    \item $\mm{B}_k$ is an $m_u \times m_u$ matrix.
\end{itemize}
%
In \autoref{eq:square-qsm-def}, the product notation is a little sloppy so, to be more explicit, this is how the products expand:
%
\begin{eqnarray}
    \prod_{k=i-1}^{j+1} \mm{A}_k &\equiv& \mm{A}_{i-1}\,\mm{A}_{i-2}\cdots\mm{A}_{j+2}\,\mm{A}_{j+1}\quad\mbox{, and} \nonumber\\
    \prod_{k=i+1}^{j-1} \mm{B}_k\T &\equiv& \mm{B}_{i+1}\T\,\mm{B}_{i+2}\T\cdots\mm{B}_{j-2}\T\,\mm{B}_{j-1}\T \quad.
\end{eqnarray}

\begin{eqnarray}
    \mm{M} &=& \left(\begin{array}{cccc}
            d_1                                & \mv{h}_1\T\mv{g}_2         & \mv{h}_1\T\mm{B}_2\T\mv{g}_3 & \mv{h}_1\T\mm{B}_2\T\mm{B}_3\T\mv{g}_4 \\
            \mv{p}_2\T\mv{q}_1                 & d_2                        & \mv{h}_2\T\mv{g}_3           & \mv{h}_2\T\mm{B}_3\T\mv{g}_4           \\
            \mv{p}_3\T\mm{A}_2\mv{q}_1         & \mv{p}_3\T \mv{q}_2        & d_3                          & \mv{h}_3\T\mv{g}_4                     \\
            \mv{p}_4\T\mm{A}_3\mm{A}_2\mv{q}_1 & \mv{p}_4\T\mm{A}_3\mv{q}_2 & \mv{p}_4\T\mv{q}_3           & d_4                                    \\
        \end{array}\right)
\end{eqnarray}

\section{The \project{celerite} matrices}

\section{Common kernel functions}

SHO, Matern, cosine, CARMA, approximations, etc., sums and products.

\section{Banded observation models}

For example, how the \project{S+LEAF} models are implemented. \citep{Delisle:2020}

\section{Multi-band time series}

With non-simultaneous observations. Generalizing \citep{Gordon:2020}.

\section{Derivative models}

Generalizing \citep{Delisle:2022}.

\section{Time-integrated models}



\begin{acknowledgments}
    DFM would like to thank the Astronomical Data Group at the Flatiron Institute for listening to every iteration of this project and for providing great feedback every step of the way.
    The authors would also like to thank
    Suzanne Aigrain (Oxford),
    Gregory Guida (Etsy),
    Christopher Stumm (Etsy),
    \emph{and others\ldots}
    for insightful conversations and other contributions.
\end{acknowledgments}

\software{showyourwork \citep{Luger:2021}, JAX \citep{Bradbury:2018}}

\appendix
\section{Algorithms}

Should also look at \citet{Pernet:2018}.

\section{Parallelization}

See \citet{Sarkka:2020}.

\bibliography{bib}
\bibliographystyle{aasjournal}

\end{document}
